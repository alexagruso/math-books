\chapter{Preliminaries}

This book assumes that the reader has had some exposure to proof writing. If
this is not the case, I highly recommend working through an introductory proof
writing book before continuing, as proofs are one of the primary methods of
communicating ideas, not only in abstract algebra, but in most higher level
mathematics.

\section*{Set Theory}

In math, a set is simply a collection of objects. These objects can be
literally anything; numbers, functions, abstract concepts, even other sets can
be elements of a set. All of the following are examples of sets:

\[
    \braces{1, 2, 3}
    \ \ \ \ 
    \braces{\text{banana}, \text{apple}, \text{pear}}
    \ \ \ \ 
    \braces{ \braces{1, 3}, \text{banana}, 1, \braces{\text{mailbox}, 5}}
\]

The order of the elements in a set
In addition, sets do not consider the ``multiplicity'' of
their elements, that is if an element is included multiple times in a set, we
view that as being equivalent to if the element was only included once. For
example, the sets \( \braces{1, 1, 2, 2} \) and \( \braces{1, 2, 2, 2, 2} \)
are both equivalent to \( \braces{1, 2} \). Sets that \textit{do} take the
multiplicity of their elements into consideration are called
\textbf{multisets}, or less commonly, \textbf{bags}. Multisets will not be used
in this book, so the reader

\subsection*{Set Builder Notation}

awd